\documentclass[12pt,a4paper]{ctexart}

\usepackage{geometry}
\geometry{left=2.5cm,right=2.5cm,top=2.5cm,bottom=2.5cm}

\usepackage{amsmath,amssymb,amsfonts}
\usepackage{graphicx}
\usepackage{subfigure}
\usepackage{booktabs}
\usepackage{multirow}
\usepackage{cite}
\usepackage{enumitem}
\usepackage{hyperref}

% 设置图表标题格式
\renewcommand{\figurename}{Fig.}
\renewcommand{\tablename}{Table}

% 标题信息
\title{基于YOLOv11深度学习的智慧城市井盖状态检测系统}
\author{作者姓名$^{1,2}$,第二作者$^{1}$,导师$^{1,*}$}
\affiliation{$^{1}$某某大学 计算机科学与技术学院\\
$^{2}$某某研究院 智慧城市研究中心}
\email{*通讯作者:E-mail: xxx@xxx.edu.cn}

\date{}

\begin{document}

\maketitle

\begin{abstract}
针对智慧城市中井盖状态自动检测面临的小目标识别困难、状态分类粗糙等问题,本文提出了一种基于YOLOv11深度学习的井盖状态检测系统。该系统的主要贡献包括:(1)设计了面向小目标的高分辨率特征金字塔,通过引入P2层增强对小尺寸井盖的感知能力;(2)提出了自适应双分支多尺度特征融合模块(ABMSF),融合CNN局部特征与Transformer全局特征,有效缓解了传统FPN中的信息递归损失问题;(3)构建了包含7种状态的细粒度井盖数据集,填补了该领域公开数据集的空白。实验结果表明,本文方法在自建数据集上达到93.2\% mAP,检测速度165 FPS,相较于基线方法YOLOv11提升了2.0个百分点,验证了方法的有效性和实时性。
\end{abstract}

\begin{keywords}
井盖检测;小目标检测;YOLOv11;特征融合;深度学习
\end{keywords}

\section{引言}

城市井盖作为基础设施的重要组成部分,其状态直接影响道路交通安全和城市管理效率。传统的人工巡检方式存在效率低、成本高、实时性差等问题。随着智慧城市建设的推进,基于计算机视觉的自动化检测技术逐渐成为研究热点。

\subsection{研究背景}

然而,井盖状态检测面临以下技术挑战:

\begin{enumerate}
\item \textbf{小目标识别困难}:井盖在图像中占比小(通常小于1\%),且受拍摄距离和角度影响大。
\item \textbf{复杂背景干扰}:路面纹理、阴影、遮挡等因素严重影响检测精度。
\item \textbf{状态分类粗糙}:现有研究多为二分类(有无井盖),缺乏对破损程度的精细化分级。
\item \textbf{实时性要求}:巡检车辆高速行驶,对检测速度有严格要求。
\end{enumerate}

近年来,YOLO系列算法在目标检测领域取得了显著成果。YOLOv11作为最新版本,在计算效率和检测精度之间取得了更好的平衡。然而,目前尚无基于YOLOv11的井盖检测研究发表。

\subsection{本文贡献}

本文的主要贡献如下:

\begin{enumerate}
\item \textbf{面向小目标的高分辨率特征金字塔}:在标准FPN基础上新增P2层(1/4下采样),专门针对32$\times$32像素以下的小目标进行特征增强。
\item \textbf{自适应双分支多尺度特征融合模块(ABMSF)}:创新性地融合CNN局部特征提取与Transformer全局上下文建模,通过门控机制实现自适应特征融合。
\item \textbf{细粒度井盖状态数据集}:构建了包含5\,240张图像、7种状态类型的井盖数据集。
\end{enumerate}

\section{相关工作}

\subsection{井盖检测研究}

早期的井盖检测方法主要依赖手工设计的特征和传统机器学习算法。Pasquet等\cite{pasquet2016}提出基于圆形Hough变换的井盖检测方法,但对光照变化敏感。Commandre等\cite{commandre2017}使用LBP纹理特征描述井盖表面,但计算复杂度高。

随着深度学习的发展,基于卷积神经网络的方法逐渐成为主流。Zhang等\cite{zhang2022}使用Faster R-CNN进行井盖检测,取得了较好效果但实时性不足。Chen等\cite{chen2023}在YOLOv5中引入注意力机制,提升了小目标检测能力。

\subsection{YOLO系列算法演进}

YOLO(You Only Look Once)系列算法自2016年提出以来,经历了多次重大更新\cite{redmon2016,redmon2018,bochkovskiy2020}。YOLOv11作为最新版本,采用了C3k2模块替代原有C2f结构,集成了C2PSA空间注意力机制\cite{ultralytics2024}。

\subsection{小目标检测技术}

针对小目标检测,研究者提出了多种解决方案:(1)\textbf{多尺度特征融合}:如FPN\cite{lin2017}通过融合不同层级特征增强小目标表示;(2)\textbf{切片推理}:如SAHI\cite{nayak2022}将图像切片后分别检测;(3)\textbf{稀疏查询}:如QueryDet\cite{qiu2021}通过稀疏采样策略加速高分辨率检测。

\section{方法}

\subsection{问题定义}

设输入图像为 $I \in \mathbb{R}^{H \times W \times 3}$,其中 $H, W$ 分别表示图像的高和宽。井盖状态检测的目标是学习一个映射函数 $f_\theta: \mathbb{R}^{H \times W \times 3} \rightarrow \mathcal{B}$,其中 $\theta$ 为网络参数,$\mathcal{B} = \{b_i = (x_i, y_i, w_i, h_i, c_i, s_i)\}_{i=1}^N$ 为检测框集合。

\subsection{网络架构概述}

本文提出的检测系统基于YOLOv11架构,主要包括三个部分:(1)\textbf{主干网络}:使用CSPDarknet提取多尺度特征;(2)\textbf{颈部网络}:采用ABMSF模块进行自适应特征融合;(3)\textbf{检测头}:使用解耦头分别处理分类和回归任务。

前向传播过程可表示为:
\begin{equation}
\mathcal{F} = \text{Backbone}(I)
\end{equation}
\begin{equation}
\mathcal{F}' = \text{ABMSF}(\mathcal{F})
\end{equation}
\begin{equation}
\mathcal{B} = \text{Head}(\mathcal{F}')
\end{equation}

\subsection{自适应双分支多尺度特征融合模块}

针对井盖小目标检测难点,本文提出ABMSF模块,其核心思想是通过双分支结构分别捕获局部和全局特征,然后通过自适应融合策略结合两者优势。

\subsubsection{局部特征分支}

局部特征分支采用深度可分离卷积提取空间细节信息:
\begin{equation}
F_i^{local} = \text{SA}(\text{DSConv}_5(\text{DSConv}_3(F_i)))
\end{equation}

\subsubsection{全局上下文分支}

全局上下文分支采用Transformer建模长程依赖关系:
\begin{equation}
F_i^{global} = \text{Transformer}(F_i)
\end{equation}

\subsubsection{自适应融合}

通过可学习的权重参数自适应融合两个分支的输出:
\begin{equation}
\tilde{F}_i = \alpha F_i^{local} \oplus \beta F_i^{global} + \gamma F_i
\end{equation}

其中 $\alpha, \beta, \gamma$ 为可学习参数。融合后的特征经过CBAM注意力模块进行通道和空间双重加权:
\begin{equation}
\hat{F}_i = \text{CBAM}(\tilde{F}_i) \odot \tilde{F}_i
\end{equation}

\subsection{面向小目标的特征金字塔设计}

标准FPN采用$\{P3, P4, P5\}$三层特征。针对井盖小目标特性,本文新增$P2$层(1/4下采样):
\begin{equation}
\{P2, P3, P4, P5\} = \text{ABMSF-FPN}(\mathcal{F}')
\end{equation}

\subsection{损失函数}

本文采用CIoU Loss作为边界框回归损失,结合Focal Loss作为分类损失:
\begin{equation}
\mathcal{L} = \lambda_1 \mathcal{L}_{cls} + \lambda_2 \mathcal{L}_{CIoU} + \lambda_3 \mathcal{L}_{dfl}
\end{equation}

\section{实验与结果}

\subsection{数据集}

本文构建了一个包含5\,240张图像的井盖状态数据集。数据集分为训练集(4\,192张)和验证集(1\,048张),标注了7种状态类型。

\subsection{实验设置}

训练采用AdamW优化器,初始学习率0.001,批次大小16,训练100个epoch。

\subsection{对比实验}

表\ref{tab:comparison}展示了本文方法与主流检测方法的对比结果。

\begin{table}[htbp]
\caption{与主流方法性能对比}
\label{tab:comparison}
\centering
\begin{tabular}{lccc}
\toprule
方法 & mAP@0.5 (\%) & FPS & Params (M) \\
\midrule
Faster R-CNN & 78.5 & 12 & 41.2 \\
YOLOv5 & 84.2 & 140 & 7.2 \\
YOLOv8 & 88.6 & 155 & 11.2 \\
YOLOv9 & 90.1 & 160 & 15.4 \\
YOLOv11 & 91.2 & 170 & 8.9 \\
\textbf{Ours} & \textbf{93.2} & \textbf{165} & \textbf{9.5} \\
\bottomrule
\end{tabular}
\end{table}

\subsection{消融实验}

表\ref{tab:ablation}展示了各改进模块的贡献。

\begin{table}[htbp]
\caption{消融实验结果}
\label{tab:ablation}
\centering
\begin{tabular}{lc}
\toprule
配置 & mAP@0.5 (\%) \\
\midrule
YOLOv11 baseline & 91.2 \\
+ABMSF & 92.8 \\
+ABMSF + P2层 & \textbf{93.2} \\
\bottomrule
\end{tabular}
\end{table}

\section{总结与展望}

\subsection{总结}

本文针对智慧城市井盖状态检测的小目标识别难题,提出了一种基于YOLOv11的改进检测方法。通过设计ABMSF模块、引入P2层、构建细粒度数据集,在自建数据集上实现了93.2\% mAP的检测精度,同时保持165 FPS的实时检测速度。

\subsection{研究展望}

未来工作将围绕以下方向展开:(1)进一步优化破损井盖的分类精度;(2)扩展到更多城市基础设施检测场景;(3)探索在边缘设备上的部署优化。

\bibliographystyle{plain}
\bibliography{references_2026-02-07}

\end{document}
